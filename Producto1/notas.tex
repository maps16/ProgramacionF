% Ejemplo de documento LaTeX. Tipo de documento y tamaño de letra
\documentclass[12pt]{article}
% Preparando para documento en Español.Para documento en Inglés no hay que hacer esto.
\usepackage[spanish]{babel}
\selectlanguage{spanish}
\usepackage[utf8]{inputenc}
% EL titulo, autor y fecha del documento
\title{Tutorial breve de los comandos de Bash}
\author{Martin Alejandro Paredes Sosa}
% Aqui comienza el cuerpo del documento
\usepackage{vmargin}
%\setpapersize{A4}
\setmargins{2.5cm}{1.5cm}{16.5cm}{23.42cm}{0pt}{1cm}{0pt}{2cm}

\begin{document}
% Construye el título
\maketitle

\section{¿Qué es {\tt bash}?}
Bash es un interpretador de comandos utilizado sobre el sistema operativo Linux. 
Su función es de mediar entre el usuario y el sistema.

\section{Navegación}
Estos comandos permiten la navegación entre los directorios. \\ \\
\begin{tabular}{|c|l|l|}
\hline
Comando & Descripción & Ejemplo \\
\hline
pwd & Muestra el directorio en el que se esta trabajando & pwd \\ \hline
ls & Muestra un lista de contenido del directorio & ls [argumento]\\ \hline
cd & Nos permite cambiar entre los directorios & cd [destino] \\\hline
which & Muestra la ruta hacia un programa en particular & which [programa] \\ \hline
\end{tabular} 

\section{Manipulacion de archivos}
Estos comando son los que te permiten manipular los archivos o directorios, como copiarlos, crearlos, borralos etre otras.\\ \\
\begin{tabular}{|c|l|l|}
\hline Comando & Descripción & Ejemplo \\ \hline
mkdir & Crea un nuevo directorio & mkdir [directorio] \\ \hline
rmdir & Remueve un directorio exixtente & rmdir [directorio] \\ \hline
touch & Crea un nuevo archivo & touch [directorio] \\ \hline
cp & Copia un archivo & cp [origen][destino] \\ \hline
mv & Mueve un archivo & mv [origen] [destino] \\ \hline
rm & Remueve un archivo & rm [directorio] \\ \hline
\end{tabular} 


\section{Trabajar con archivos}
Estos comandos te permiten realizar visualizar cierta información de los archivos. \\
\begin{tabular}{|c|l|l|} 
\hline
Comando & Descripción & Ejemplo \\ \hline
cat & Te muestra el contenido de un archivo & cat [archivo] \\ \hline
vi & Te permite editar un archivo & vi [archivo] \\ \hline
less & Te permite moverte dentro de un archivo & less [archivo]\\ \hline
head & Te permite visualizar la parte superior del escrito & head [archivo]  \\ \hline
tail & Te permite visualizar la parte inferior del escrito & tail [archivo]  \\ \hline
sort & organiza la información del archivo & sort [archivo] \\ \hline
nl & Te enumera los reglones del texto & nl [archivo] \\ \hline
wc & Muestra numero de lineas, palabras y caracteres & wc [archivo] \\ \hline
sed & Busca y Remplaza en el archivo & sed [arg]/[bsc]/[remp]/[archivo] \\ \hline
uniq & remueve lineas iguales & uniq [archivo] \\ \hline
tac & Muestra la información en orden inverso & tac [archivo]\\ \hline
\end{tabular}

\section{MISC}
Existen diferentes tipos de comandos, unos te permiten moverte por los directorios, otros manipular tu archivos. Pero tambien existen otra variedad de comandos que nos permiten realizar otras cosas, como el manejo de los procesos y permisos.
\\ \begin{tabular}{|c|l|l|}
\hline
Comando & Descripción & Ejemplo \\
\hline
man[comando] & Te muestra el uso del comando & man ls: mustra contenido \\ \hline
echo & Muestra el mensaje que devuelve la variable & echo [variable] \\ \hline
chmod & Cambia los permisos de un directorio & chmod [permisos] [archivo] \\ \hline
egrep & Muestra lineas que coinciden con un patrón & egrep [patrón] [archivo] \\ \hline
top & Ver los proceso que se estan ejecutando & top \\ \hline
ps & Muestra los procesos en ejecución & ps [aux] \\ \hline
kill & Termina un proceso & kill [proceso] \\ \hline
jobs & Muestra los procesos en segundo plano & jobs \\ \hline
fg & Mueve procesos a primer plano & fg [num proceso] \\ \hline
\end{tabular} 


% Nunca debe faltar esta última linea.
\end{document}
